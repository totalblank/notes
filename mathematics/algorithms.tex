\input{../preamble.tex}

\title{Introduction to algorithm - notes}
\author{Aheer Srabon}
\date{}

\begin{document}
\maketitle

\section{Divide and conquer}
\noindent \textbf{Divide} the problem into one or more subproblems that are smaller instances
of the same problem.

\noindent \textbf{Conquer} the subproblems by solving them recursively.

\noindent \textbf{Combine} the subproblem solutions to form a solution to the
original problem.

\subsection{Recurrences}
\noindent A \emph{recurrence} is an equation that describes a function in terms of its
value on other, typically smaller, arguments. 
\begin{itemize}
	\item \emph{Recursive case} - if a case involves the recursvie invocation of the
		function on different (usually smaller) inputs.
	\item \emph{Base case} - if it's not a recursive case.
\end{itemize} 

\noindent The recurrence is \emph{well-defined} if there is at least one function that 
satisfies it. \emph{Ill-defined} otherwise.

\noindent A recurrence $ T(n) $ is \emph{algorithmic} if, for every sufficiently large
threshold costant $ n_0 > 0 $, the following two properties hold:
\begin{enumerate}
	\item For all $ n < n_0 $, we have $ T(n) = \Theta (1) $.
	\item For all $ n \geq n_0 $, every path of recursion terminates in a defined base
		case within a finite number of recursive invocations.
\end{enumerate}

\subsection{Solving recurrences}
\begin{itemize}
	\item \emph{Substitution method}
	\item \emph{Recursion-tree method}
	\item \emph{Master method}
	\item \emph{Akra-Bazzi method}
\end{itemize}

\end{document}
