%%%%%%%%%%%%%%%%%%%%%%%%%%%%% Define Article %%%%%%%%%%%%%%%%%%%%%%%%%%%%%%%%%%
\documentclass{article}
%%%%%%%%%%%%%%%%%%%%%%%%%%%%%%%%%%%%%%%%%%%%%%%%%%%%%%%%%%%%%%%%%%%%%%%%%%%%%%%

%%%%%%%%%%%%%%%%%%%%%%%%%%%%% Using Packages %%%%%%%%%%%%%%%%%%%%%%%%%%%%%%%%%%
\usepackage{geometry}
\usepackage{graphicx}
\usepackage{amssymb}
\usepackage{amsmath}
\usepackage{amsthm}
\usepackage{empheq}
\usepackage{mdframed}
\usepackage{booktabs}
\usepackage{lipsum}
\usepackage{graphicx}
\usepackage{color}
\usepackage{psfrag}
\usepackage{pgfplots}
\usepackage{bm}
\usepackage{hyperref}
\usepackage{array}
%%%%%%%%%%%%%%%%%%%%%%%%%%%%%%%%%%%%%%%%%%%%%%%%%%%%%%%%%%%%%%%%%%%%%%%%%%%%%%%

% Other Settings

%%%%%%%%%%%%%%%%%%%%%%%%%% Page Setting %%%%%%%%%%%%%%%%%%%%%%%%%%%%%%%%%%%%%%%
\geometry{a4paper}

%%%%%%%%%%%%%%%%%%%%%%%%%% Define some useful colors %%%%%%%%%%%%%%%%%%%%%%%%%%
\definecolor{ocre}{RGB}{243,102,25}
\definecolor{mygray}{RGB}{243,243,244}
\definecolor{deepGreen}{RGB}{26,111,0}
\definecolor{shallowGreen}{RGB}{235,255,255}
\definecolor{deepBlue}{RGB}{61,124,222}
\definecolor{shallowBlue}{RGB}{235,249,255}
%%%%%%%%%%%%%%%%%%%%%%%%%%%%%%%%%%%%%%%%%%%%%%%%%%%%%%%%%%%%%%%%%%%%%%%%%%%%%%%

%%%%%%%%%%%%%%%%%%%%%%%%%% Define an orangebox command %%%%%%%%%%%%%%%%%%%%%%%%
\newcommand\orangebox[1]{\fcolorbox{ocre}{mygray}{\hspace{1em}#1\hspace{1em}}}
%%%%%%%%%%%%%%%%%%%%%%%%%%%%%%%%%%%%%%%%%%%%%%%%%%%%%%%%%%%%%%%%%%%%%%%%%%%%%%%

%%%%%%%%%%%%%%%%%%%%%%%%%%%% English Environments %%%%%%%%%%%%%%%%%%%%%%%%%%%%%
\newtheoremstyle{mytheoremstyle}{3pt}{3pt}{\normalfont}{0cm}{\rmfamily\bfseries}{}{1em}{{\color{black}\thmname{#1}~\thmnumber{#2}}\thmnote{\,--\,#3}}
\newtheoremstyle{myproblemstyle}{3pt}{3pt}{\normalfont}{0cm}{\rmfamily\bfseries}{}{1em}{{\color{black}\thmname{#1}~\thmnumber{#2}}\thmnote{\,--\,#3}}
\theoremstyle{mytheoremstyle}
\newmdtheoremenv[linewidth=1pt,backgroundcolor=shallowGreen,linecolor=deepGreen,leftmargin=0pt,innerleftmargin=20pt,innerrightmargin=20pt,]{theorem}{Theorem}[section]
\theoremstyle{mytheoremstyle}
\newmdtheoremenv[linewidth=1pt,backgroundcolor=shallowBlue,linecolor=deepBlue,leftmargin=0pt,innerleftmargin=20pt,innerrightmargin=20pt,]{definition}{Definition}[section]
\theoremstyle{myproblemstyle}
\newmdtheoremenv[linecolor=black,leftmargin=0pt,innerleftmargin=10pt,innerrightmargin=10pt,]{problem}{Problem}[section]
%%%%%%%%%%%%%%%%%%%%%%%%%%%%%%%%%%%%%%%%%%%%%%%%%%%%%%%%%%%%%%%%%%%%%%%%%%%%%%%

%%%%%%%%%%%%%%%%%%%%%%%%%%%%%%% Plotting Settings %%%%%%%%%%%%%%%%%%%%%%%%%%%%%
\usepgfplotslibrary{colorbrewer}
\pgfplotsset{width=8cm,compat=1.9}
%%%%%%%%%%%%%%%%%%%%%%%%%%%%%%%%%%%%%%%%%%%%%%%%%%%%%%%%%%%%%%%%%%%%%%%%%%%%%%%

% Define new commands
\newcommand{\innerp}[2]{\left\langle #1 \vert #2 \right\rangle}

%%%%%%%%%%%%%%%%%%%%%%%%%%%%%%% Title & Author %%%%%%%%%%%%%%%%%%%%%%%%%%%%%%%%
\title{Linear Partial Differential Equations}
\author{Aheer Srabon}
%%%%%%%%%%%%%%%%%%%%%%%%%%%%%%%%%%%%%%%%%%%%%%%%%%%%%%%%%%%%%%%%%%%%%%%%%%%%%%%

\begin{document}
    \maketitle

    \section{Syllabus}
    \subsection{Description}
    \noindent This course provides students with the basic analytical and computational tools 
    of linear partial differential equations (PDEs) for practical applications in science 
    engineering, including heat/diffusion, wave, and Poisson equations.

    \noindent Analytics emphasize the viewpoint of linear algebra and the analogy with finite matrix
    problems including operator adjoints and eigenproblems, series solutions, Green’s functions,
    and separation of variables.

    \noindent Numerics focus on finite-difference and finite-element techniques to reduce PDEs to matrix
    problems, including stability and convergence analysis and implicit/explicit time-stepping.

    \noindent Julia programming language is introduced and used in homework for simple examples.
    Julia is a high-level, high-performance dynamic language for technical computing,
    with syntax that is familiar to users of other technical computing environments.
    It provides a sophisticated compiler, distributed parallel execution, numerical accuracy,
    and an extensive mathematical function library.

    \subsection{Recommended books}
    \begin{itemize}
    	\item Computational Science and Engineering by 
		\href{https://math.mit.edu/~gs/cse/}{Strang}, Gilbert.
	\item Introduction to Partial Differential Equations by Olver, Peter.
    \end{itemize}
    \subsection{Lecture plan}
    \begin{itemize}
    	\item Overview of linear PDEs and analogies with matrix algebra
	\item Poisson's equation and eignefunctions in 1D: Fourier
		sine series
	\item Finite difference methods and accuracy
	\item Discrete vs. continuous Laplacians: Symmetry and dot products
	\item Diagonalizability of infinite-dimensional Hermitian operators
	\item Start with a truly discrete (finite dimensional) system,
		and then derive the continuum PDE model as a limit or
		approximation
	\item Start in 1D with the "Sturm-Liouville operator", generalize
		Sturm-Liouville operators to multiple dimensions
	\item Music and wave equations, separation of variables, in time
		and space
	\item Separation of variables in cylindrical geometries: Bessel
		functions
	\item General Dirichlet and Neumann boundary conditions
	\item Multidimensional finite differences
	\item Kronecker products
	\item The min-max theorem
	\item Green's functions with Dirichlet boundaries
	\item Reciprocity and positivity of Green's functions
	\item Delta functions and distributions
	\item Green's function of $ \Delta^2 $ in 3D for infinite space,
		the method of images
	\item The method of images, interfaces, and surface integral
		equations
	\item Green's functions in inhomogeneous media: Integral equations
		and Born approximations
	\item Dipole sources and approximations, Overview of time-dependent
		problems
	\item Time-stepping and stability: Definitions, Lax equivalence
	\item Von Neumann analysis and the heat equation
	\item Algebraic properties of wave equations and unitary time
		evolution, Conservation of energy in a stretched string
	\item Staggered discretizations of wave equations
	\item Traveling waves: D'Alembert's solution
	\item Group-velocity derivation and dispersion.
	\item Material dispersion and convolutions
	\item General topic of waveguides, Superposition of modes,
		Evanscent modes
	\item Waveguide modes, Reduced eigenproblem
	\item Guidance, reflection, and refraction at interfaces between
		regions with different wave speeds
	\item Numerical examples of total internal reflection
	\item Perfectly matched layers (PML)
	\item Perturbation theory and Hellman-Feynman theorem
	\item Finite element methods: Introduction
	\item Galerkin discretization
	\item Convergence proof for the finite-element method, Boundary
		conditions and the finite-element method
	\item Finite-element software
	\item Symmetry and linear PDEs
    \end{itemize}

    \section{Lecture notes}
    \subsection{L 01 - Overview of linear PDEs and analogies with matrix algebra}

    \noindent A few important PDEs are,
    \begin{itemize}
    	\item Poisson's equation
	\item Laplace's equation
	\item Heat/diffusion equation
	\item Scalar wave equation
    \end{itemize}

    \noindent and many many others ...
    \begin{itemize}
	\item Maxwell (electromagnetism)
	\item Navier-Stokes (fluids)
	\item Schrodinger (quantum mechanics)
	\item Black-Scholes
    \end{itemize}

    \begin{center}
	\begin{tabular}{ | m{5cm} | m{5cm} | m{5cm} | }
		\hline
		 & constant coefficients = 1 &
		 variable coefficients = c(\textbf{x}) \\

		 \hline
		 Poisson's equation: 
		 &
		 $ \Delta^2 u = f $ \newline \newline
		 \textbf{example:} $ f $ = charge density, 
		 \newline
		 $ u $ = - electric potential 
		 \newline
		 \textbf{example:} $ f $ = heat source / sink rate 
		 \newline
		 $ u $ = steady-state temperature 
		 \newline
		 \textbf{example:} $ f $ = solute source / sink rate 
		 \newline
		 $ u $ = steady state concentration 
		 \newline
		 \textbf{example:} $ f  \sim $ force on stretched string/drum 
		 \newline
		 $ u $ = steady-state displacement
		 &
		 $ \Delta \cdot (c \Delta u) = f $
		 \newline \newline \newline
		 c = permittivity $ \varepsilon $ \newline \newline \newline
		 c = thermal conductivity \newline \newline \newline
		 c = diffusion coefficient \newline \newline \newline
		 $ c \sim $ "springy-new" \\

		 \hline
		 Laplace's equation:
		 &
		 $ \Delta^2 u = 0 $ \newline
		 \textbf{examples:} as for Poisson, but no sources
		 &
		 $ \Delta \cdot (c \Delta u) = 0 $ \\

		 \hline
		 Heat / diffusion equation:
		 &
		 $ \frac{\partial u}{\partial t} = \Delta^2 u $ \newline
		 \textbf{examples:} $ u $ = temperature \newline
		 $ u $ = solute concentration
		 &
		$ \frac{\partial^2 u}{\partial t^2} = \Delta \cdot (c \Delta u) $ \newline
		$ c $ = thermal conductivity \newline
		$ c $ = diffusion coefficient \\

		 \hline
		 Scalar wave equation:
		 &
		 $ \frac{\partial^2 u}{\partial t^2} = \Delta^2 u $ \newline
		 \textbf{examples:} \newline
		 $ u $ = displacement of stretched string / drum \newline
		 $ u $= density of gas / fluid 
		 &
		 $ \frac{\partial^2 u}{\partial t^2} = \Delta \cdot (c \Delta u) $ \newline
		 $ c^2 = $ 1 / wave speed \\

		 \hline

    	\end{tabular}
    \end{center}

    \noindent Difference between 18.06 and 18.303 is outlined. As I didn't
    take 18.06, I wouldn't know. 

    \subsection{L 02 - Poisson's equation and eignefunctions in 1D: Fourier sine series}
    \subsubsection{Orthogonality of two functions}
    \noindent Two real functions $ f(x) $ and $ g(x) $ are orthogonal if for any interval 
    $ (a, b) $ in both of their domain, the following is true,
    \begin{equation}
	    \int_{a}^{b} f(x) g(x) \, dx = 0
	    \label{eqn:1}
    \end{equation}
    
    \noindent The dot product of two functions is defined as,
    \begin{equation}
	f(x) \cdot g(x) = \int_{-\infty}^{\infty} f(x) g(x) \, dx
    	\label{eq:2}
    \end{equation}

    \noindent If the functions are complex functions, the dot product (or inner product)
    of the two functions are defined as,

    \begin{equation}
	\innerp{f}{g} = \int_{-\infty}^{-\infty} \overline{f(x)} g(x) \, dx
    	\label{eq:3}
    \end{equation}

    \noindent Equations \ref{eq:2} and \ref{eq:3} need not to be from $ -\infty $ to $ \infty $.
    The limit can be any interval that is being considered.
    It can be shown than for two sine function $ sin(n \pi x) $ and 
    $ sin(m \pi x) $, they are orthogonal if $ n \neq m $. They are not orthogonal
    otherwise.

    \subsubsection{Fourier sine series}
    \noindent The Fourier sine series for a function $ f(x) $ defined on
    $ x \in [0,1] $ writes $ f(x) $ as,

    \begin{equation}
	    f(x) = \sum_{n=1}^{\infty} b_n sin(n \pi x)
    	\label{eq:4}
    \end{equation}

    \noindent $ b_n $ can be computed by multiplying both sides of equation \ref{eq:4}
    by $ sin(m \pi x) $ and then integrating both sides from 0 to 1.

    \begin{equation}
	b_m = 2 \int_{0}^{1} f(x) sin(m \pi x) \, dx
    	\label{eq:5}
    \end{equation}

    \noindent Fourier claimed (without proof) that \emph{any} function
    $ f(x) $ can be expanded in terms of sines in this way, even
    discontinuous functions! That is, these sine functions form an
    \emph{orthogonal basis} for "all" functions! But this turned out to be
    false. Rather Carleson (1966) and Hunt (1968) proved that: any function
    $ f(x) $ where $ \int (\lvert f(x) \rvert)^p \, dx $ is finite for some
    $ p > 1 $ has a Fourier series that converges \emph{almost} everywhere
    to $ f(x) $ (except at isolated points). At points where $ f(x) $ has
    a jump discontinuity, the Fourier series converges to the midpoint of
    the jump. Except for crazy \emph{divergent functions} or the function
    values exactly at points of discontinuity, Fourier's remarkable claim
    is essentially true.

    \noindent for $ f(x) = 1 $,
    \begin{equation}
	b_n = 
    	\begin{cases}
		\frac{4}{n \pi} & \text{n odd} \\
		0 & \text{n even}
    	\end{cases}
    	\label{eq:6}
    \end{equation}

    \pagebreak

    \noindent For triangular function $ f(x) = \frac{1}{2} - \lvert x - \frac{1}{2} \rvert $, $ b_{m, \text{odd}} = \frac{4}{(m \pi)^2} (-1)^{\frac{m-1}{2}} $ . And the Fourier sine series of that triangular function is,

    \begin{equation}
	    f(x) = \frac{4}{\pi^2} sin(\pi x) - \frac{4}{(3 \pi)^2} sin(3 \pi x) + \frac{4}{(5 \pi)^2} sin(5 \pi x) + ...
    	\label{eq:7}
    \end{equation}

\end{document}
