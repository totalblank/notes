%%%%%%%%%%%%%%%%%%%%%%%%%%%%% Define Article %%%%%%%%%%%%%%%%%%%%%%%%%%%%%%%%%%
\documentclass{article}
%%%%%%%%%%%%%%%%%%%%%%%%%%%%%%%%%%%%%%%%%%%%%%%%%%%%%%%%%%%%%%%%%%%%%%%%%%%%%%%

%%%%%%%%%%%%%%%%%%%%%%%%%%%%% Using Packages %%%%%%%%%%%%%%%%%%%%%%%%%%%%%%%%%%
\usepackage{geometry}
\usepackage{graphicx}
\usepackage{caption}
\usepackage{subcaption}
\usepackage{amssymb}
\usepackage{amsmath}
\usepackage{amsthm}
\usepackage{empheq}
\usepackage{mdframed}
\usepackage{booktabs}
\usepackage{lipsum}
\usepackage{graphicx}
\usepackage{color}
\usepackage{psfrag}
\usepackage{pgfplots}
\usepackage{bm}
\usepackage{float}
\usepackage{physics}
\usepackage{hyperref}
\usepackage{xcolor}
\usepackage{titlesec}
\usepackage{algorithmic}
\hypersetup{
	colorlinks=true,
	linkcolor=blue,
	filecolor=magenta,
	urlcolor=cyan
}
\usepackage{animate}
\usepackage{svg}
% Quantum physics inner product
\newcommand{\innerp}[2]{\left\langle #1 \vert #2 \right\rangle}
\newcommand{\convolution}[2]{(#1 \star #2)}

%%%  \begin{figure}[htp]
%%%  	\centering
%%%  	\includegraphics[width=0.25\textwidth]{/path/to/image}
%%%  	\caption{}\label{fig:}
%%%  \end{figure}

% to write multiple lines of equations

%%  \begin{align*}
%%  \begin{split}
%%  	\ket{\phi} + \ket{\psi} &= \ket{\phi + \psi} \\
%%  	a \ket{\psi} &= \ket{a \psi}
%%  \end{split}
%%  \end{align*}

\newcommand{\e}[1]{\times 10^{#1}}

\usepackage{listings}
\lstset{showstringspaces=false} % pretty embeded code in LaTeX
%%%%%%%%%%%%%%%%%%%%%%%%%%%%%%%%%%%%%%%%%%%%%%%%%%%%%%%%%%%%%%%%%%%%%%%%%%%%%%%

% Other Settings

%%%%%%%%%%%%%%%%%%%%%%%%%% Page Setting %%%%%%%%%%%%%%%%%%%%%%%%%%%%%%%%%%%%%%%
\geometry{a4paper}

%%%%%%%%%%%%%%%%%%%%%%%%%% Define some useful colors %%%%%%%%%%%%%%%%%%%%%%%%%%
\definecolor{ocre}{RGB}{243,102,25}
\definecolor{mygray}{RGB}{243,243,244}
\definecolor{deepGreen}{RGB}{26,111,0}
\definecolor{shallowGreen}{RGB}{235,255,255}
\definecolor{deepBlue}{RGB}{61,124,222}
\definecolor{shallowBlue}{RGB}{235,249,255}
%%%%%%%%%%%%%%%%%%%%%%%%%%%%%%%%%%%%%%%%%%%%%%%%%%%%%%%%%%%%%%%%%%%%%%%%%%%%%%%

%%%%%%%%%%%%%%%%%%%%%%%%%% Define an orangebox command %%%%%%%%%%%%%%%%%%%%%%%%
\newcommand\orangebox[1]{\fcolorbox{ocre}{mygray}{\hspace{1em}#1\hspace{1em}}}
%%%%%%%%%%%%%%%%%%%%%%%%%%%%%%%%%%%%%%%%%%%%%%%%%%%%%%%%%%%%%%%%%%%%%%%%%%%%%%%

%%%%%%%%%%%%%%%%%%%%%%%%%%%% English Environments %%%%%%%%%%%%%%%%%%%%%%%%%%%%%
\newtheoremstyle{mytheoremstyle}{3pt}{3pt}{\normalfont}{0cm}{\rmfamily\bfseries}{}{1em}{{\color{black}\thmname{#1}~\thmnumber{#2}}\thmnote{\,--\,#3}}
\newtheoremstyle{myproblemstyle}{3pt}{3pt}{\normalfont}{0cm}{\rmfamily\bfseries}{}{1em}{{\color{black}\thmname{#1}~\thmnumber{#2}}\thmnote{\,--\,#3}}
\theoremstyle{mytheoremstyle}
\newmdtheoremenv[linewidth=1pt,backgroundcolor=shallowGreen,linecolor=deepGreen,leftmargin=0pt,innerleftmargin=20pt,innerrightmargin=20pt,]{theorem}{Theorem}[section]
\theoremstyle{mytheoremstyle}
\newmdtheoremenv[linewidth=1pt,backgroundcolor=shallowBlue,linecolor=deepBlue,leftmargin=0pt,innerleftmargin=20pt,innerrightmargin=20pt,]{definition}{Definition}[section]
\theoremstyle{myproblemstyle}
\newmdtheoremenv[linecolor=black,leftmargin=0pt,innerleftmargin=10pt,innerrightmargin=10pt,]{problem}{Problem}[section]
%%%%%%%%%%%%%%%%%%%%%%%%%%%%%%%%%%%%%%%%%%%%%%%%%%%%%%%%%%%%%%%%%%%%%%%%%%%%%%%

%%%%%%%%%%%%%%%%%%%%%%%%%%%%%%% Plotting Settings %%%%%%%%%%%%%%%%%%%%%%%%%%%%%
\usepgfplotslibrary{colorbrewer}
\pgfplotsset{width=8cm,compat=1.9}
%%%%%%%%%%%%%%%%%%%%%%%%%%%%%%%%%%%%%%%%%%%%%%%%%%%%%%%%%%%%%%%%%%%%%%%%%%%%%%%

%% Centering section titles
\titleformat{\section}[block]{\Large\bfseries\filcenter}{}{1em}{}

%% norm
%%\newcommand\norm[1]{\lVert #1 \rVert}


\title{Quantum Physics}
\author{Aheer srabon}
\date{}

\begin{document}
\maketitle

	\section{Linear algebra}
	\noindent Physical quantities can sometimes be discrete,
	not only continuous.

	\section{Kets and wave function}
	\noindent  Linear algebra and vector spaces are all about
	structure and patterns, not about the type of objects being
	used. A particle is represented by a vector in a vector space.
	A vector in this vector space represent a quantum state. A
	quantum state is a mathematical object that holds all the
	physical properties of a particle.

	\noindent Quantum superposition,
	\begin{equation}
		\ket{\psi} = \sum c_i \ket{E_i}
		\label{eq:1}
	\end{equation}

	\noindent Here, $ \ket{\psi} $ is the quantum state of the 
	particle, which is a linear combination of the states (
	$ \ket{E_i} $. The number of states can be infinite. This issue
	is solved by \emph{Hilbert space} ($ \mathcal{H} $).

	\noindent Although some positions can be more or less likely,
	position, as a whole, is not discrete (as far as we know).
	Let's say, the quantum state of a particle is the linear combination
	of position kets. As position is continuous, the linear combination
	is the integral from $ -\infty $ to $ +\infty $.

	\begin{equation}
		\ket{\psi} = \int_{-\infty}^{+\infty} dx \cdot \psi(x) \cdot \ket{x}
		\label{eq:2}
	\end{equation}

	\noindent So, a wavefunction is a continuous list of coefficients whenever
	the list of kets in infinite.

	\section{Hilbert space}
	\noindent We know that, quantum state of a particle is the linear combination
	of its states where the coefficient function is the wave function of the
	particle.

	\begin{equation}
		\ket{\psi} = \sum_{i=1}^{n} c_i \cdot \ket{E_i}
		\label{eq:3}
	\end{equation}

	\noindent But this $ n $ can be infinite as well. If $ n $
	is infinite, things get problematic.

	\subsection{Problem with infinite sum}
	\noindent Let's consider the basis vector to be the set of all polynomials
	$ {1, x, x^2, x^3,\dots} $. Now, let's consider an infinite linear combination
	of elements of this set (i.e., infinite linear combination of the basis
	vectors), $ 1 + x + \frac{x^2}{2!} + \frac{x^3}{3!} + \dots $, which clearly
	is $ e^x $.

	\begin{equation}
		e^x = 1 + x + \frac{x^2}{2!} + \frac{x^3}{3!} + \dots
		\label{eq:4}
	\end{equation}

	\noindent But the problem is, this $ e^x $ is outside of the set of basis
	vectors that constitute it. So, infinitely combining the basis vectors linearly
	gives us something outside of the set of basis vectors. So, if our quantum
	state ($ \ket{\psi(x)} $) is an infinite superposition of outcome states
	($ \{\ket{E_i}\} $ or $ \{\ket{P_i}\} $), then there's a chance that this quantum state is
	outside of our vector space, therefore, not a quantum state at all. 

	\subsection{Solution}

	\noindent Let's apply an extra following rule to our vector space,
	\begin{center}
		\emph{Every convergent sum of vectors must converge to an element inside
		of the vector space.}
	\end{center}

	\noindent If this rule is applied (along with all the rules of a vector space), then
	the resulting space is called a \emph{Hilbert} space $ \mathcal{H} $. The rigorous 
	definition of a \emph{Hilbert} space is,

	\begin{center}
		\emph{\textbf{Hilbert Space: } A vector space equipped with an \emph{inner product}
		that is \emph{Cauchy complete}}.
	\end{center}

	\noindent Cauchy completeness (or complete metric space) is defined as follows,

	\begin{center}
		\emph{Every convergent sequence (or sum) of vectors (e.g., partial sums
		of infinite linear combination) converges to an element inside of the
		vector space}
	\end{center}

	\noindent So,
	\begin{equation}
		\ket{\psi} \in \mathcal{H}
		\label{eq:5}
	\end{equation}

	\noindent For the time being, inner product can be thought of as being the dot
	product.

	\section{Inner product}
	\noindent At its core, an \emph{inner product} takes in two vectors and outputs
	a number which might be complex,
	\begin{equation}
		\innerp{\psi}{\phi} = c\notag
	\end{equation}
	\noindent $ \innerp{\psi}{\phi} $ is the inner product of kets. There are
	a few rules for kets,
	
	\begin{align*}
	\begin{split}
		\ket{\phi} + \ket{\psi} &= \ket{\phi + \psi} \\
		a \ket{\psi} &= \ket{a \psi} \\
		\innerp{\psi}{\phi + \zeta} &= \innerp{\psi}{\phi} + \innerp{\psi}{\zeta} \\
		\innerp{\phi}{a \psi} &= a \innerp{\phi}{\psi} 
	\end{split}
	\end{align*}

	\noindent The last two rules state that, the inner product should be linear in the
	right hand position. Magnitude of a ket is $ \abs{\innerp{\phi}{\phi}} $, and it is
	\emph{positive}.

	\noindent All the rules stated above are intuitive from dot product. But
	the \emph{commutative} law does not hold for the inner product, which can be proved
	through contradiction. 

	\noindent Let's assume that the inner product is indeed commutative and the magnitude
	is positive.
	
	\begin{align*}
		\innerp{i\phi}{i\phi} = i\innerp{i\phi}{\phi} = i\innerp{\phi}{i\phi} \\
			= i^2 \innerp{\phi}{\phi} = - \innerp{\phi}{\phi}
	\end{align*}
	\noindent Which results in the magnitude of the ket $ \abs{\ket{i \phi}} = i \abs{\ket{\phi}} $. This
	cannot happen (apparantly, the magnitude should not be complex). That is why, the
	inner product is not communtative. Rather the following rule applies,

	\begin{align*}
		\innerp{\psi}{\phi} = \innerp{\phi}{\psi}^*
	\end{align*}

	\noindent which solves the previous problem.

	\vspace{0.5cm}

	\noindent One final rule is, only the zero vector has zero length. So, the inner product
	definition is as follows,

	\begin{center}
		\emph{An inner product $ \innerp{\psi}{\phi} $ is a map from vectors
		to scalars which satisfies the following rules,}
	\end{center}

	\begin{align*}
	\begin{split}
		\innerp{\psi}{\zeta + \phi} &= \innerp{\psi}{\zeta} + \innerp{\psi}{\phi} \\
		\innerp{\psi}{a \phi} &= a \innerp{\psi}{\phi} \\
		\innerp{\psi}{\phi} &= \innerp{\phi}{\psi}^* \\
		\text{for} \ket{\psi} &\neq 0, \innerp{\psi}{\psi} > 0 \\
	\end{split}
	\end{align*}

\end{document}
